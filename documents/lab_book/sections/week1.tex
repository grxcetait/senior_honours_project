
\section{Week 1: 12th - 18th January 2026}

\begin{itemize}
  \item Before first week of the semester, emailed Zhaoting to set up a meeting for the first week back to get started with the project.
  \item Got sent links to a Python package to install and the documentations to set up a Python environment.
  \begin{itemize}
      \item Python package: https://github.com/zhaotingchen/meer21cm 
      \item Documentation: https://meer21cm.readthedocs.io/en/latest/
  \end{itemize}
\end{itemize}

\subsection{Monday 12th January 2026}

\begin{itemize}
    \item Installed Python package "meer21cm" and set up a Python environment
    \item Note: before running code make sure to activate environment: conda activate meer21cm
\end{itemize}

\subsection{Tuesday 13th January 2026}

\subsubsection{Meeting 1: Zhaoting and Alkistis}

\begin{itemize}
    \item Went over the basic ideas of the project and the end goal. I am writing what I remember from the conversation so these following points may be incorrect.
    \begin{itemize}
        \item Can take measurements of the Universe at different frequencies and map this in 3D.
        \item Universe is H abundant so we can map out intensity maps of H which can show us where there are galaxies.
        \item This is important with cosmological redshift.
        \item Most radiation is synchrotron radiation from the Milky Way which is smooth.
    \end{itemize}
\item Importance of the report
\begin{itemize}
    \item Will be read by two markers who are not necessary informed on this type of astrophysics so I need to make sure my background is very well done and it explains things very well and highlights the significance of why this is important.
\end{itemize}
\item Went over the rough timeline of the project
\begin{itemize}
    \item Make sure to balance getting the results and writing the report
\end{itemize}
\item If I am stuck of have any questions, make sure to ask!
\item Planned weekly meetings for Thursday at 11am
\item For meeting 2:
\begin{itemize}
    \item Run some of the files from the package in "Cookbook", not the Data Analysis Recipes but the Theory Recipes
    \item Read the papers in the project description
    \begin{itemize}
        \item In "GPR for foreground removal in HI intensity mapping experiments": important to understand how top and bottom of Figure 1 related and how Figure 1 and Figure 2 relates.
        \item Understand the maths and some of the derivations?
        \item Understand PCA
    \end{itemize}
\item Will be send Jupyter notebooks later in the week which will be used for the project
\end{itemize}
\end{itemize}