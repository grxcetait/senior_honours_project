\section{Week 2: 19th - 25 January 2026}

\subsection{Monday 19th January 2026}

Between meeting 1 last Tuesday and meeting 2 this Thursday, I want to have completed the tasks set by Zhaoting and have set up everything I need for this project so workflow is seamless. This will take a while to set up initially (because I am not completely familiar with GitHub) but will serve me well when I am doing my final report write-up and doing my Masters project next year.

\subsubsection{Plan for this week before Thursday meeting:}

\begin{itemize}
    \item Read over papers and make notes on anything unfamiliar
    \end{itemize}
    \begin{itemize}
    \item Run cookbook files
    \item Complete notebook 1 task
    \item Set up GitHub repository
    \item Start writing in lab book
    \end{itemize}

\subsection{Tuesday 20th January 2026}

\begin{itemize}
\item Read over the 3 papers from the project description and made notes on anything unfamiliar.
\end{itemize}
\begin{itemize}
    \item So far, only read abstracts and introductions. I've been highlighting and annotating the downloaded copy on GoodNotes.
    \item Need to still read conclusions and everything else and also write down summaries of the papers.
\end{itemize}
\begin{itemize}
    \item Started to set up a GitHub repository
        \begin{itemize}
            \item Created repository
        \end{itemize}
\end{itemize}
\begin{itemize}
    \begin{itemize}
        \item Cloned it to terminal
    \end{itemize}
\end{itemize}
        \begin{itemize}
            \begin{itemize}
                \item Linked Overleaf account and created document and sections for lab book (so any outputs from code is easily accessible for Overleaf and I don't have to keep downloading images etc. and everything is in one place)
            \end{itemize}
        \end{itemize}
        \begin{itemize}
            \begin{itemize}
                \item Set up folders for code week by week (eg. notebook 1, ...), another folder for code outputs (eg. week 1, ...), and another folder for academic papers
            \end{itemize}
        \end{itemize}
        \begin{itemize}
            \begin{itemize}
                \item Downloaded Zotero (for easy referencing) but still having linked this to my GitHub!
            \end{itemize}
        \end{itemize}

\subsection{Wednesday 21st January 2026}

\color{red}To-do:
\begin{itemize}
    \item GitHub: 
    \begin{itemize}
        \item Link Zotero
    \end{itemize}
    \begin{itemize}
        \item Add a folder for any academic papers I find
    \end{itemize}
    \item Papers: read conclusions and everything else of the papers
    \item Cookbook code: write summaries of what they do
    \item Notebook 1 task: summary of what I had to do and image of output
\end{itemize}

\color{black}\begin{itemize}
\item Finished setting up my GitHub repository
\end{itemize}
\begin{itemize}
        \item I organised folders, etc.
        \item I added a rough README.md
    \end{itemize}
    \item Ran cookbook codes
    \begin{itemize}
        \item Location: /Users/gracetait/shproject/meer21cm/docs/source/cookbook
    \end{itemize}
    \begin{itemize}
        \item Found the theory and data analysis code. Didn't run the data analysis code because I don't have any data to run yet and didn't run the mock recipes either
    \end{itemize}
    \begin{itemize}
        \item Ran theory recipes. There was four in total
\end{itemize}
\item Ran notebook 1 code and completed task 1

\color{black}\subsubsection{Running cookbook codes}

\begin{itemize}
    \item Effect of weights in the power spectrum estimator and the modelling
    \item On the importance of having the correct survey dimension and k-sampling for intensity mapping
    \item Mode-mixing due to PCA
    \item Adding observational effects in your model power spectrum
\end{itemize}

